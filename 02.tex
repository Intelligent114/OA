\documentclass[cn,hazy,cyan,11pt,normal]{elegantnote}
\title{最优化算法作业1}

\author{陈文轩}

\date{\today}

\usepackage{array}
\usepackage{amsfonts}
\usepackage{bbm}
\usepackage{mathtools}
\usepackage{extarrows}




\definecolor{c1}{HTML}{017eda}
\definecolor{c2}{HTML}{1f1e33}

\begin{document}

    \maketitle

    \begin{enumerate}
        \item {\color{c1}考虑以下问题:
            \begin{flalign*}
               \min_{x\in\mathbb{R}^2}\quad& x_1^2+x_2^2\\
               s.t.\quad& (x_1-1)^2+(x_2-1)^2\leq 2\\
               &(x_1-1)^2+(x_2+1)^2\leq 2
            \end{flalign*}
            其中$x=(x_1,x_2)^{\top}\in\mathbb{R}^2$。
            \begin{enumerate}
                \item 这是一个凸优化问题吗?
                \item 写出此问题的拉格朗日函数。使用Slater条件验证在这个问题中是否存在强对偶性。
                \item 写出这个优化问题的KKT条件。求出KKT点和最优点。
            \end{enumerate}}

            \vspace{0.5cm}\textcolor{c2}解

            \begin{enumerate}
                \item 由于目标函数与约束函数均为二次函数,且Hessian矩阵为$2I_2$,故是凸优化问题。
                \item 引入非负Lagrange乘子$\lambda_1,\lambda_2\geq0$,可以得到问题的Lagrange函数为:
                    \[L(x,\lambda_1,\lambda_2)=x_1^2+x_2^2+\lambda_1((x_1-1)^2+(x_2-1)^2-2)+\lambda_2(x_1-1)^2+(x_2+1)^2-2\]

                    取$x=(1,0)^{\top}$,显然这个$x$满足严格不等式,故Slater条件成立,即具有强对偶性。
                \item KKT条件为:\vspace{0.2cm}

                    $\begin{cases}
                        \nabla_x L(x,\lambda_1,\lambda_2)=2(x_1+(\lambda_1+\lambda_2)(x_1-1),x_2+(\lambda_1+\lambda_2)(x_2-1))=0  \\
                        \lambda_1((x_1-1)^2+(x_2-1)^2-2)=0,\lambda_2((x_1-1)^2+(x_2+1)^2-2)=0  \\
                        (x_1-1)^2+(x_2-1)^2\leq 2,(x_1-1)^2+(x_2+1)^2\leq 2 \\
                        \lambda_1\geq0,\lambda_2\geq0
                    \end{cases}$

                    \vspace{0.2cm}解得$x_1=x_2=\lambda_1=\lambda_2=0$,即$(0,0)^{\top}$是KKT点,也是最优点。
            \end{enumerate}

        \item {\color{c1}\vspace{0.5cm}考虑一个凸的分段线性最小化问题,变量为$x\in\mathbb{R}^n$
            \[\min\max_{i=1,\cdots,m}(a^{\top}_i x+b_i)\]
            其中$a_i\in\mathbb{R}^n,n_i\in\mathbb{R}$。

            \begin{enumerate}
                \item 考虑原问题的如下等价问题\ref{eq:eq1},推导\ref{eq:eq1}的对偶问题
                    \begin{equation}
                        \begin{aligned}
                        \min\max_{i=1,\cdots,m}\quad& y_i\\
                        s.t.\quad& a^{\top}_i x+b_i\leq y_i,i=1,\cdots,m\\
                        \end{aligned}
                        \tag{2.1}
                        \label{eq:eq1}
                    \end{equation}
                    变量为$x\in\mathbb{R}^n,y\in\mathbb{R}^m$。
                \item 假设我们通过平滑函数逼近目标函数, 逼近函数为:
                    \[f_0(x)=\log\sum_{i=1}^m \exp(a_i^{\top}x+b_i)\]
                    现在我们考虑无约束问题,即$\min f_0(x)$。证明该问题的对偶问题如下:
                    \begin{equation}
                        \begin{aligned}
                        \max\quad& b^{\top}z-\sum_{i=1}^m z_i\log z_i\\
                        s.t.\quad& A^{\top}z=0,\mathbbm{1}^{\top}z=1,z\succeq0.\\
                        \end{aligned}
                        \tag{2.2}
                        \label{eq:eq2}
                    \end{equation}
                    其中$\mathbbm{1}$表示全是$1$的向量。
                \item 设问题\ref{eq:eq1}的最优函数值是$p_1^*$,问题\ref{eq:eq2}的最优函数值是$p_2^*$,证明$0\leq p_2^*-p_1^*\leq \log m$。
            \end{enumerate}}

            \vspace{0.5cm}\textcolor{c2}解

            \begin{enumerate}

                \item 先转化为线性规划问题:
                    \begin{flalign*}
                        \min\quad& t\\
                        s.t.\quad& a^{\top}_i x+b_i\leq t,i=1,\cdots,m
                    \end{flalign*}
                    Lagrange函数为$L(x,t,\lambda)=t+\sum\limits_{i=1}^m \lambda_i(a^{\top}_i x+b_i-t),\lambda_i\geq0$,目标是$\max\limits_{\lambda}\min\limits_{x,t}L(x,t,\lambda)$。
                    $\nabla_t L(x,t,\lambda)=1-\sum\limits_{i=1}^m \lambda_i=0\Rightarrow \mathbbm{1}^{\top}\lambda=1,\nabla_x L(x,t,\lambda)=\sum\limits_{i=1}^m \lambda_i a_i=0$,

                    此时$L(x,y,\lambda)=\sum\limits_{i=1}^m \lambda_i b_i$。因此对偶问题是:
                    \begin{flalign*}
                        \max\quad& \sum_{i=1}^m \lambda_i b_i\\
                        s.t.\quad& \mathbbm{1}^{\top}\lambda=1,\sum_{i=1}^m \lambda_i a_i=0\\
                        &\lambda_i\geq0,i=1,\cdots,m
                    \end{flalign*}

                \item 记$A=(a_i,\cdots,a_m),z_j=\dfrac{\exp(a_j^{\top}x+b_j)}{\sum_{i=1}^m \exp(a_i^{\top}x+b_i)}$,则$\sum\limits_{j=1}^m z_j=1,z_j\geq0$。

                    此时$f_0(x)=\log\sum\limits_{i=1}^m \exp(a_i^{\top}x+b_i)=\sum\limits_{i=1}^m z_i(a_i^{\top}x+b_i)-\sum\limits_{i=1}^m z_i\log z_i\coloneqq L(x,z)$。

                    $\nabla_x L(x,z)=\sum\limits_{i=1}^m z_i a_i=0\Rightarrow A^{\top}z=0$,此时$L(x,z)=b^{\top}z-\sum\limits_{i=1}^m z_i\log z_i$。

                    因此对偶问题是:
                    \begin{flalign*}
                        \max\quad& b^{\top}z-\sum_{i=1}^m z_i\log z_i\\
                        s.t.\quad& A^{\top}z=0,\mathbbm{1}^{\top}z=1,z\succeq0.
                    \end{flalign*}

                \item 记$M=\max\limits_i (a^{\top}_i x+b_i)$,则$\exp(M)\leq\sum\limits_{i=1}^m \exp(a_i^{\top}x+b_i)\leq m\exp(M)$。

                    取对数,即有$M\leq\log\sum\limits_{i=1}^m \exp(a_i^{\top}x+b_i)\leq M+\log m,\forall x$。

                    对问题\ref{eq:eq2}的最优解$x_2^*$,有$p_2^*=\log\sum\limits_{i=1}^m \exp(a_i^{\top}x_2^*+b_i)$,由上有$\max\limits_i(a_i^{\top}x_2^*+b_i)\leq p_2^*$。

                    又$p_1^*=\min\limits_x \max\limits_i (a_i^{\top}x+b_i)\leq\max\limits_i(a_i^{\top}x_2^*+b_i)\leq p_2^*$,故$p_1\leq p_2$。

                    对问题\ref{eq:eq1}的最优解$x_1^*$,有$\max\limits_i(a_i^{\top}x_1^*+b_i)=p_1^*$。由上有$\log\sum\limits_{i=1}^m \exp(a_i^{\top}x_1^*+b_i)\leq p_1^*+\log m$。又$p_2^*=\min\limits_x \log\sum\limits_{i=1}^m \exp(a_i^{\top}x+b_i)\leq p_1^*+\log m$,故$p_2^*\leq p_1^*+\log m$。

                    综上所述有$0\leq p_2^*-p_1^*\leq \log m$。
            \end{enumerate}

        \item {\color{c1}\vspace{0.5cm}若$f(x)=\| x\|$表示任意范数,$x\in\mathbb{R}^n$,证明次微分集合如下:
            \[\partial f(x)=\{g\in\mathbb{R}^n:\|g\|_*\leq1,\langle g,x\rangle=\|x\|\},\]
            其中$\|g\|_*$表示对偶范数。}

            \vspace{0.5cm}\textcolor{c2}解

            由次微分定义,有$g\in\partial f(x)\Rightarrow f(y)\geq f(x)+\langle g,y-x\rangle,\forall y$,即$\|y\|\geq \|x\|+\langle g,y-x\rangle,\forall y$。

            分以下情况讨论:
            \begin{itemize}
                \item 若$x=0$,则$\langle g,x\rangle=\|x\|$显然成立;

                \item 对$x\neq0$,取$y=0$,则$0\geq\|x\|+\langle g,-x \rangle\Leftrightarrow\langle g,x \rangle\geq\|x\|$;

                    $\qquad\qquad\,\,\,$取$y=2x$,则$2\|x\|\geq\|x\|+\langle g,x \rangle\Leftrightarrow\langle g,x \rangle\leq\|x\|$。故$\langle g,x \rangle=\|x\|$。


            \end{itemize}
            而$g\in\partial f(x)\Rightarrow \|y\|\geq\langle g,y\rangle,\forall y$。令$v=\dfrac{y}{\|y\|},t=\|y\|$,则条件变为:

            $t\|v\|\geq t\langle g,v\rangle\Leftrightarrow \langle g,v\rangle\leq1,\forall \|v\|=1$。由对偶范数定义,这等价于$\|g\|_*=\sup\limits_{\|v\|=1}\langle g,v \rangle\leq1$。

            故$\{g\in\mathbb{R}^n: \|g\|_*\leq1,\langle g,x\rangle=\|x\|\}\subset\partial f(x)$。

            反之,若$g\in\mathbb{R}^n$满足$\|g\|_*\leq1,\langle g,x \rangle=\|x\|$,

            则$\|y\|\geq\|y\|\cdot\|g\|_*\geq\langle g,y \rangle=\langle g,y-x \rangle+\langle g,x \rangle=\|x\|+\langle g,y-x \rangle,\forall y\Rightarrow g\in\partial f(x)$。

            故$\partial f(x)\subset\{g\in\mathbb{R}^n: \|g\|_*\leq1,\langle g,x\rangle=\|x\|\}\Rightarrow\partial f(x)=\{g\in\mathbb{R}^n:\|g\|_*\leq1,\langle g,x\rangle=\|x\|\}$。

        \item {\color{c1}\vspace{0.5cm}对于$y\in\mathbb{R}^m$,给定$\mu-$强凸,$L-$光滑函数$g(y)$,即$\nabla g(y)$的Lipchitz常数为$L$。若$A\in\mathbb{R}^{m\times n},m\leq n$是行满秩矩阵,证明:

            \begin{enumerate}
                \item $f(x)=g(Ax)$是$\bar{L}-$光滑的,其中
                    \[\bar{L}=L\|A\|^2\]
                    $\|A\|$表示矩阵$A$的谱范数。

                \item $f(x)=g(Ax)$满足规则性条件,即
                    \[\langle \nabla f(x),x-x_{\text{proj}} \rangle\geq\bar{\mu}\|x-x_{\text{proj}}\|^2\]
                    其中$x_{\text{proj}}$表示$x$到函数$f(x)$最小值解集合的正交投影点,$\bar{\mu}=\mu\lambda_{\min}(AA^{\top})$。
            \end{enumerate}}

            \vspace{0.5cm}\textcolor{c2}解

            \begin{enumerate}
                \item 由于$\nabla f(x)=A^{\top}\nabla g(Ax)$,有:
                    \begin{flalign*}
                        \|\nabla f(x_1)-\nabla f(x_2)\|&=\|A^{\top}\nabla g(Ax_1)-A^{\top}\nabla g(Ax_2)\| \leq\|A^{\top}\| \cdot\|\nabla g(x_1)-\nabla g(x_2)\| &\\
                                                       &\leq \|A^{\top}\|\cdot L\cdot \|Ax_1-Ax_2\|\leq \|A^{\top}\|\cdot L\cdot \|A\|\cdot\|x_1-x_2\| &\\
                                                       &=\|A\|^2\cdot L\cdot\|x_1-x_2\|\coloneqq\bar{L}\|x_1-x_2\|
                    \end{flalign*}
                    故$f(x)$是$\bar{L}-$光滑的。
                \item 由$g(y)\mu-$强凸,有$g(y)$有唯一的最小值点$y^*$,且$\nabla f(y^*)=0$。$A$行满秩$\Rightarrow AA^{\top}\succ 0$。

                    此时$\mu\|y-y^*\|^2\le\langle \nabla g(y)-\nabla g(y^*),y-y^* \rangle=\langle \nabla g(Ax),Ax-y^* \rangle $。

                    又$\forall v\in \mathrm{Row}(A),\|Av\|^2\geq \lambda_{\min}(AA^{\top})\|v\|^2$,$x-x_{\text{proj}}\in \mathrm{Col}(A^{\top})=\mathrm{Row}(A)$
                    \begin{flalign*}
                        \langle \nabla f(x),x-x_{\text{proj}} \rangle&=\langle A^{\top}\nabla g(Ax),x-x_{\text{proj}} \rangle=\langle \nabla g(Ax),A(x-x_{\text{proj}})\rangle &\\
                        &=\langle \nabla g(Ax),Ax-y^*) \rangle\geq \mu\|Ax-y^*\|^2=\mu\|A(x-x_{\text{proj}})\|^2 &\\
                        &\geq\mu\lambda_{\min}(AA^{\top})\|x-x_{\text{proj}}\|^2\coloneqq \bar{\mu}\|x-x_{\text{proj}}\|^2 &
                    \end{flalign*}
                    故$\langle \nabla f(x),x-x_{\text{proj}} \rangle\geq\bar{\mu}\|x-x_{\text{proj}}\|^2$。
            \end{enumerate}

        \item {\color{c1}\vspace{0.5cm}考虑凸函数$f(x)$的共轭函数
            \[f^*(y)=\sup_x(x^{\top}y-f(x))\]
            证明:
            \begin{enumerate}
                \item 若$x\in\partial f(y)$,则$y\in\partial f^*(x)$;
                \item 若$f(x)$是闭凸函数,利用$f=f^{**}$证明$x\in\partial f(y)$等价于$y\in\partial f^*(x)$。
            \end{enumerate}}

    \end{enumerate}


\end{document}